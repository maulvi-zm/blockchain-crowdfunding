% !TeX root = ../main.tex
\section{Related Work}

Crowdfunding has been widely adopted through centralized online platforms such as
Kickstarter and GoFundMe, which provide end-to-end services for campaign creation,
donation processing, and fund distribution \cite{b5,b6}. These platforms operate under
a trusted intermediary model, where campaign validation, fund custody, and refund
decisions are controlled by the platform operator. While this approach simplifies user
experience and scalability, it introduces fundamental limitations related to transparency,
auditability, and single points of failure. Contributors must rely on the platform to
correctly enforce campaign rules and refund policies, particularly in cases of failed or
disputed campaigns.

From an academic perspective, the dynamics and challenges of centralized crowdfunding
have been studied extensively. Mollick analyzed behavioral and structural aspects of
crowdfunding platforms, highlighting the importance of trust and information asymmetry
between contributors and campaign creators \cite{b1}. However, such studies generally
assume centralized governance and do not address technical enforcement of trust through
decentralized mechanisms.

To mitigate these limitations, blockchain-based crowdfunding systems have been proposed
as an alternative. Several works demonstrate the use of smart contracts to encode
crowdfunding logic, such as funding goals, deadlines, and conditional fund release, directly
on-chain, enabling transparent and immutable execution \cite{b2}. These approaches reduce
reliance on centralized intermediaries and ensure that contributors can independently
verify campaign outcomes. Nevertheless, many existing implementations focus primarily on
on-chain logic and do not sufficiently address usability, off-chain integration, or failure
handling, which limits their applicability in real-world scenarios.

Another relevant research direction concerns the integration of off-chain data into
blockchain systems through oracle mechanisms. Zhang et al.\ discussed architectural
constraints of smart contracts and emphasized the necessity of external data sources for
practical applications \cite{b3}. Caldarelli further analyzed the oracle problem, showing
that oracle-based designs introduce new risks such as data manipulation, delayed responses,
and availability failures \cite{b4}. While oracle mechanisms have been widely studied in
domains such as decentralized finance, their systematic application in crowdfunding,
particularly for handling campaign finalization and adverse scenarios, remains limited.

In contrast to prior work, this project combines insights from centralized crowdfunding
platforms with decentralized smart contract execution and oracle-based off-chain
integration. The proposed system emphasizes transparent campaign lifecycle management,
deterministic refund logic, and robust handling of off-chain failures, thereby addressing
both trust and practicality concerns in blockchain-based crowdfunding applications.
