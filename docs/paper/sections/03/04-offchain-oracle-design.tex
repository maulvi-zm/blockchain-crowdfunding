% !TeX root = ../../main.tex

\subsection{Off-chain Oracle Design}
\subsubsection{Oracle Workflow}
The oracle mechanism enables the smart contract to incorporate external data
sources in a trust-controlled manner. When off-chain data is required, the smart
contract emits an \texttt{OracleDataRequested} event containing the campaign
identifier, a unique request identifier, a data key specifying the requested
information, and an auxiliary parameter.

The oracle service operates according to the following workflow:
\begin{enumerate}[label=(\roman*)]
  \item Subscribes to on-chain \texttt{OracleDataRequested} events.
  \item Queries the corresponding external HTTP API endpoint.
  \item Validates and normalizes the returned data (e.g., fixed-point scaling).
  \item Submits the result on-chain via the \texttt{oracleCallback} function,
        including the request identifier, data key, normalized value, and update
        timestamp, as a signed transaction.
\end{enumerate}

\subsubsection{API Data Structure}
The oracle queries an external HTTP API whose specification is summarized in Table~\ref{tab:oracle-api}. As shown in Listing~\ref{lst:oracle-json}, the response contains the currency pair, the exchange rate, and a UNIX timestamp.
\begin{table}[t]
\centering
\caption{Oracle API specification}
\label{tab:oracle-api}
\begin{tabular}{@{}ll@{}}
\toprule
\textbf{Field} & \textbf{Value} \\
\midrule
Method & GET \\
Endpoint & \texttt{/rate} \\
Query parameter & \texttt{pair = ETH\_IDR} \\
Response format & JSON \\
\bottomrule
\end{tabular}
\end{table}
\begin{lstlisting}[style=jsonstyle, caption={Oracle API JSON response}, label={lst:oracle-json}]
{
  "pair": "ETH_IDR",
  "rate": 56000000.12,
  "timestamp": 1769990000
}
\end{lstlisting}
The oracle normalizes the floating-point exchange rate into a fixed-point integer before submitting it on-chain. Specifically,
\begin{equation}
v = \lfloor r \times 10^{2} \rfloor,
\end{equation}
where $r$ denotes the exchange rate returned by the API. The timestamp is propagated as the \texttt{updatedAt} field in the callback transaction.

\subsubsection{Failure Handling and Oracle Manipulation Mitigation}
To handle external API failures and oracle delays:
\begin{itemize}
  \item Retry with backoff: bounded retries (e.g., 3 attempts) before marking a request unresolved.
  \item Staleness checks: the contract rejects updates older than a threshold or ignores stale values at finalize time.
  \item Single fulfillment: \texttt{requestId} is marked fulfilled to prevent replay.
  \item Trust boundary: only a whitelisted oracle address may call \texttt{oracleCallback}.
\end{itemize}
These measures reduce risk from oracle downtime and manipulation attempts.
