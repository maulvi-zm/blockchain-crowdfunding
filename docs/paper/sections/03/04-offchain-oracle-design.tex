% !TeX root = ../../main.tex

\subsection{Off-chain Oracle Design}
\subsubsection{Oracle Workflow}
The oracle mechanism enables the smart contract to incorporate external data. The contract emits \texttt{OracleDataRequested(campaignId, requestId, dataKey, param)} and the oracle service:
\begin{enumerate}
  \item Subscribes to \texttt{OracleDataRequested} events.
  \item Queries an external API endpoint (HTTP).
  \item Validates and normalizes data (e.g., fixed-point scaling).
  \item Submits \texttt{oracleCallback(campaignId, requestId, dataKey, value, updatedAt)} as a signed transaction.
\end{enumerate}

\subsubsection{API Data Structure}
The paper documents the API endpoint and JSON structure. Example exchange-rate usage:
\begin{itemize}
  \item Endpoint: \texttt{GET /rate?pair=ETH\_IDR}
  \item JSON payload:
\end{itemize}
\begin{verbatim}
{
  "pair": "ETH_IDR",
  "rate": 56000000.12,
  "timestamp": 1769990000
}
\end{verbatim}
The oracle converts \texttt{rate} to a fixed-point integer (e.g., \texttt{value = rate * 100}) and sets \texttt{updatedAt} to the API timestamp.

\subsubsection{Failure Handling and Oracle Manipulation Mitigation}
To handle external API failures and oracle delays:
\begin{itemize}
  \item \textbf{Retry with backoff:} bounded retries (e.g., 3 attempts) before marking a request unresolved.
  \item \textbf{Staleness checks:} the contract rejects updates older than a threshold or ignores stale values at finalize time.
  \item \textbf{Single fulfillment:} \texttt{requestId} is marked fulfilled to prevent replay.
  \item \textbf{Trust boundary:} only a whitelisted oracle address may call \texttt{oracleCallback}.
\end{itemize}
These measures reduce risk from oracle downtime and manipulation attempts.
