
\subsection{On-chain Smart Contract Design}
\subsubsection{State Model}
Each campaign is represented by an immutable creator identity and a set of
mutable state variables, including the funding goal, deadline, total amount
raised, campaign status, withdrawal flag, and an off-chain metadata content
identifier (CID). The campaign lifecycle is defined by a small, finite set of
states, with deterministic transitions governed by the campaign deadline and
funding outcome. The core states are summarized in Table~\ref{tab:campaign_states}, and the transition logic is depicted in Fig.~\ref{fig:campaign-lifecycle}.

\begin{table}[htbp]
\centering
\caption{Campaign Lifecycle States}
\label{tab:campaign_states}
\begin{tabularx}{\columnwidth}{@{\extracolsep{\fill}}p{0.28\columnwidth}X@{}}
    \toprule
    \textbf{State} & \textbf{Definition} \\
    \midrule
    ACTIVE & Accepts contributions until the campaign deadline. \\
    SUCCESS & Finalized after the deadline if the raised amount meets or exceeds
    the goal; allows creator withdrawal. \\
    FAILED & Finalized after the deadline if the raised amount is below the goal;
    allows contributor refunds. \\
    \bottomrule
\end{tabularx}
\end{table}

\begin{figure}[htbp]
  \centering
  \includegraphics[width=0.65\columnwidth]{../images/campaign-lifecycle-state-machine.png}
  \caption{Campaign lifecycle state machine showing finalize conditions and post-finalization actions.}
  \label{fig:campaign-lifecycle}
\end{figure}

\subsubsection{Core Functions and Authorization}
Authorization is enforced at the contract level, and each entry point is scoped
to a minimal role. Campaigns are created via \texttt{createCampaign}, which
initializes campaign parameters and emits a \texttt{CampaignCreated} event.
Contributions are accepted through \texttt{contribute}, emitting
\texttt{ContributionReceived} upon receipt of funds. After the campaign deadline,
\texttt{finalizeCampaign} deterministically transitions the campaign from the
\texttt{ACTIVE} state to either \texttt{SUCCESS} or \texttt{FAILED}, emitting
\texttt{CampaignFinalized}.

Fund flows are strictly restricted. The \texttt{withdrawFunds} function is
callable only by the campaign creator and emits \texttt{FundsWithdrawn}, while
\texttt{refund} is callable only by contributors with a positive recorded
contribution and emits \texttt{RefundIssued}. To preserve composability and
support off-chain indexing, external systems rely on emitted events as the
canonical log of state transitions.

\subsubsection{Security Considerations}
Withdrawal and refund execution paths follow the checks-effects-interactions
pattern and are additionally protected by reentrancy guards. Contributor
iteration is deliberately avoided on-chain; instead, refunds are implemented as
pull-based operations initiated by individual contributors. Oracle callbacks
are restricted to a pre-authorized oracle address to prevent unauthorized state
manipulation.
